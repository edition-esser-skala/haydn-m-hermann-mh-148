\documentclass{ees}

\begin{document}

\eesTitlePage

\eesCriticalReport{
  –    & –   & bc    & Bass figures only appear in movement 4·4.
                       All other bass figures have been added
                       by the editor. \\
  1·12 & 27f & vla   & each 1st \halfNote\ in \B1: c+c′4–\crotchetRest \\
       & –   & –     & \B1 contains the following directive below
                       the final staff, relating to the final bar:
                       “NB l’ultima Nota \breveNote\fermata\ col arco
                       fortissimo e decrescendo in pianissimo”. \\
  2·7  & 7   & vla   & 1st \quarterNote\ in \B1: g′4 \\
  2·8  & 10  & bc    & grace note missing in \B1 \\
       & 24  & vla   & 1st \quarterNote\ in \B1: g4 \\
  2·11 & 57  & ob, cor & bar in \B1: \wholeNoteRest \\
       & 57  & vl 2  & bar in \B1: 2 × \crotchetRest \\
  2·12 & 32  & cor 2 & bar missing in \B1 \\
  3·1  & 5   & cor   & rhythm of 2nd \halfNote\ in \B1: \halfNote \\
       & 39  & soli  & 3rd \eighthNote\ in \B1: b16–b16 \\
  3·2  & 53  & vl 2  & 1st \eighthNote\ in \B1: \sharp c″8 \\
       & 222 & Montanus & 1st \halfNote\ in \B1: a′4–\crotchetRest \\
  3·3  & 1   & –     & The first bar of this movement is also printed
                       at the end of the previous movement to accomodate
                       the overlapping vocal parts. \\
  3·4  & 30  & –     & \B1 lacks a \textit{fine} mark; however,
                       the 1st \quarterNote\ of bar 30 represents
                       the most sensitive way to end the movement. \\
  3·5  & 16  & Venus & last \eighthNote\ missing in \B1 \\
  3·6  & 102 & ob 2  & bar missing in \B1 \\
  3·8  & 62  & vla   & grace note missing in \B1 \\
  3·9  & 5   & vla   & 7th to 12th \sixteenthNote\ in \B1: 6 × e′16 \\
       & 34  & ob    & bar missing in \B1 \\
       & 96  & ob    & last \eighthNote\ in \B1: d″8 \\
  4·2  & –   & –     & In \B1, this movement is split into two movements:
                       The first one comprises piff, cor, and tamb,
                       while the second one comprises the remaining
                       instruments. In addition, the directive
                       “NB voce umana cogli Pifferi”
                       appears above the movement. \\
  4·4  & 16  & manicordo & 4th \sixteenthNote\ in \B1: \sharp f′16 \\
}

\eesToc{}

\eesScore

\end{document}

3.1

Teutogenus
Quid video?
Quae me terra, quae regio bonis opibusque
condecorata naturae optimis,
quae pulcra, quae fecunda me tellus habet?
Temporibus his unde venit adflictis decor?
Quis providus adest pastor,
et custos loci, qui didicit,
inter temporum et rerum arduas patriam procellas tegere,
ut antiqui nihil splendoris illa perdat? –
An vivit novus Pompilius?
An Saturnus in terram aurea saecula reduxit?
Iste Thessaliam meis oculis adumbrat fertilem et pulcram locus.
Adeste, quisquis nobiles colitis plagas,
nomenque me docete felicis soli.

Montanus
Quis novus hic hospes?
Nosse quis Salzam cupit?

Teutogenes
Amice!

Montanus
Quid Teutogene! quid montes petis,
qui plana lambis littora,
et valles amas.

Teutogenes
Descende! nil morare, nil metuas mali!

Montanus
Parebo votis.
Promptus en jussa exsequor: –
Adsum. Petitam numquid annonam advehis?

Teutogenes
Inanis ad te venio; nam Cereris nihil huc
adtulisse Marte tam dubio licet.

Montanus
Heu me! Peribit subditus, segetes nisi propitius
hac multiplicet in terra Deus.

Teutogenes
Hem! quid lacessis astra?
Quid numen pium querulis fatigas precibus,
et metuis famen, dum vix beatior exstat in terris locus,
ac iste, cui natura materna manu videtur indidisse divitias simul,
quas arctiore dextera multis declit?
Amice!
Si fortasse te miserum auctumas alias
revise lumine attento plagas:
et tunc videbis, quanta tibi coelum bona contulevit.
Arctus undique est panis:
tibi natura subministrat insolitas opes.

Montanus
Id ego paterna Principis curae optimi adscribo,
qui pro subditi vigilat bono,
teneroque nos prosequitur affectu.

Teutogenes
Istius mihi pande nomen Principis,
ut illum colam.

Montanus
Videbis ex hoc monte percusso iubar,
quod exhibebit Principis nostri decus,
scutumque tectum crinibus perspicuae aquae.

Teutogenes
Quae mira nobis decora de saxo elicis.
Novi fluentes rivuli nitidas aquas,
quae puritatem Principis vestri notant.
Novi micantis igneas Aellae faces
quibus indicatur fervidus in omnes amor:
Novi virentes ramuli teneras comas,
queis firma spes in optimum innuitur Deum:
Novi coruscam clypei et in medio crucem,
qua viva designatur in superos fides:
Novi patentes Patris in patriam manus,
qui sublevare subditos opibus cupit,
et liberali dextera miseros iuvat.
Novi Leonis vigilis impavidam indolem. –

Montanus
Exasse iam te Principem nosse optimum intelligo.
Ecce! Sicut e saxo decus gentile fulget Principis,
sic et meo in corde nomen ipsius scriptum nitet.
Quod serus, et venturus aliquando nepos
post plura per me saecula extolli audiet.
En! musa nostra cogitat gratam sui amoris obtulisse tesseram,
sed tam pio quid obtulisse Principi acceptum queat?

Teutogenes
Tu convenire magna scis magnis:
mea hinc si sequare consilia,
magna elige, et tunc placebit musa,
ne dubites, tua.

Montanus
Magna petis, at magna simul,
ut placeam edoce.

Teutogenes
Primatis Archipraesul in nostris tuus nomen habetoris.
Sede ab hac prima fides divina luxit Teutoni,
qui se prius vinclis ligatum daemonis doluit miser.
Hinc musa pro Primate Primatem canet!
Vis nosse, Primas iste quis fuerit?
Pius Hermannus est, Teutonia quem merito colit.
Hic triste primus omnium excussit iugum,
quo Roma capti Teutonis pressit caput.
Nobilius invenisse vix aliud queas exemplar,
in Primate Primatem colens.

Montanus
Amice!
Tua consilia praeprimis placent.
Tuas cruore soepius caesi suo,
varia quirites strage tinxerunt aquas:
musam ergo nostram facta, quae nosti edoce.

Teutogenes
A fratre Rheno plurima edoctus scio:
Hinc longiores nectere haud opus est moras.
Accede carum Principem, et patriae patrem,
Teutonia quem submissa Primatem dolit.
Accinge digno strenuas operi manus,
meamque spera rebus in dubiis opem. –

Montanus
Excelse Princeps! Columen et patriae decus!
En musa tibi devota, quae Salzam bibit,
hodie laboris citius abrepti quidem,
dedisse specimen gestit.
Hermanni impigros pro patria atque subditis,
quod tu facis, hodie labores breviter in scenam dabit.
Ignosce Princeps nostra si forsan lyra,
non in cothurni purpura, et versu gravi incedat;
ornamenta nam nimium breve surripuit
ista tempus et praeceps labor.
Hinc si canentis sermo displiceat,
bona placeat voluntas:
Musa, quod potuit, agit.
Sigmunde! Clemens adspice Hermannum, et fave.


3.2

Teutogenes
Felix Germania, felix!
cui mira providentia
haec sors beata obtigit
Hermannum nutriendi.

Montanus
Felix Iuvavia, felix!
cui Numinis clementia
haec sors beata obtigit
Sigmundum eligendi.

both
In Te, o Princeps! integra
Germania laetatur.

Teutogenes
Hunc principem belliducem

Montanus
Te Principem Antistitem

both
ut solem inter sidera
suspensa demiratur.


3.4

Venus
Sein reizendes Weſen, ſein artiger Blick
ſchlägt Jupiters raſſelnde Pfeile zurück.

Bacchus

Die ſtrotzende Rebe mit munterem Saft
giebt Göttern den Nektar und Menſchen die Kraft.

both
Mars ſelbſt wird bezwungen durch Liebe und Wein:
mir hat es gelungen ſein Herrſcher zu seyn.



3.5

both
Es kann alſo Bacchus und Venus allein
der mächtigſte Kriegsgott der Sterblichen ſeyn.
Bey unſeren Gaben wird Herkules ſchwach,
es geben auch Tiger und Panterthier nach.
Hier ſind die Beſieger vom Schlafe betäubt:
bis ein deütſcher Krieger die Schwachen entleibt.


3·6

Chorus Bardorum, Chorus Ducum et Militum
Magne Deus Teutonum
audi vota subplicum!
Fac ut per victoriam
liberemus patriam.

Per Tuisci merita
da virtutis praemia,
per Gomeris clypeum
tege vultus Teutonum.

Qui ditasti maxime
patres nostrae patriae,
da felicem exitum
post devictum praelium.
